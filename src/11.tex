\begin{exercise}
	Let $F$ be the vector field $F(x,y) = (x^3 + xy^2, x^2 y - y^5)$.
	\begin{enumerate}
		\item Let $C$ be the square curve with corner points $(0,0), (1,0), (1,1), (0,1)$ and anti-clockwise orientation. Determine $\int_C F \cdot dr$.

		\item Show that $F$ has a potential $\phi$. How can you explain the result of part (a)?

		\item Compute the line integral of $F$ for a curve from $(0,0)$ to $(1,2)$.
	\end{enumerate}
\end{exercise}

\begin{solution}
	\begin{enumerate}
		\item
		      % \begin{enumerate}[label=\arabic*.]
		      %  \item $C_1$: from $(0,0)$ to $(1,0)$ (bottom edge):
		      %        $$
		      %         C_1: r(t) = (x(t),y(t)) = (t,0), \; 0 \le t \le 1
		      %        $$
		      %        $dx = x'(t)dt = dt$, $dy = y'(t)dt = 0$, $y = 0$:
		      %        $$
		      %         \begin{aligned}
		      %          \int_{C_1} F \cdot dr = \int_{0}^{1} P\,dx + Q\,dy & = \int_{0}^{1} P\,dx                                   \\
		      %                                                             & = \int_{0}^{1} \left( t^{3} + t \cdot 0^2 \right) \,dt \\
		      %                                                             & = \int_{0}^{1} t^{3}\,dt = \frac{1}{4}
		      %         \end{aligned}
		      %        $$
		      %  \item $C_2$: from $(1,0)$ to $(1,1)$ (right edge)
		      %        $$
		      %         C_2: r(t) = (x(t),y(t)) = (1,t), \; 0 \le t \le 1
		      %        $$
		      %        $dx = 0$, $dy = dt$, $x = 1$:
		      %        $$
		      %         \begin{aligned}
		      %          \int_{C_2} \F \cdot dr = \int_{0}^{1} P\,dx + Q\,dy & = \int_{0}^{1} Q\,dy                                                         \\
		      %                                                              & = \int_{0}^{1} (t - t^{5})\,dt                                               \\
		      %                                                              & = \left[ \tfrac{1}{2}t^{2} - \tfrac{1}{6}t^{6} \right]_{0}^{1} = \frac{1}{3}
		      %         \end{aligned}
		      %        $$
		      %
		      %  \item $C_3$: from $(1,1)$ to $(0,1)$ (top edge):
		      %        $$
		      %         C_3: r(t) = (x(t),y(t)) = (1-t,1), \; 0 \le t \le 1
		      %        $$
		      %        $dx = -dt$, $dy = 0$, $y = 1$:
		      %        $$
		      %         \begin{aligned}
		      %          \int_{C_3} F \cdot dr = \int_{0}^{1} P\,dx + Q\,dy & = \int_{0}^{1} P\,dx                                                                        \\
		      %                                                             & = \int_{0}^{1} \left( (1-t)^{3} + (1-t) \right) \, (-dt)                                    \\
		      %                                                             & = -\int_{0}^{1} (1-t)^{3}\,dt - \int_{0}^{1} (1-t)\,dt                                      \\
		      %                                                             & = -\left[ -\tfrac{1}{4}(1-t)^4 \right]_{0}^{1} - \left[ t - \tfrac{1}{2}t^2 \right]_{0}^{1} \\
		      %                                                             & = -\frac{1}{4} - \frac{1}{2} = -\frac{3}{4}
		      %         \end{aligned}
		      %        $$
		      %
		      %  \item $C_4$: from $(0,1)$ to $(0,0)$ (left edge):
		      %        $$
		      %         C_4: r(t) = (x(t),y(t)) = (0,1-t), \; 0 \le t \le 1
		      %        $$
		      %        $dx = 0$, $dy = -dt$, $x = 0$:
		      %        $$
		      %         \begin{aligned}
		      %          \int_{C_4} F \cdot dr  = \int_{0}^{1} P\,dx + Q\,dy & = \int_{0}^{1} Q\,dy                                        \\
		      %                                                              & = \int_{0}^{1} \left( -(1-t)^{5} \right) \, (-dt)           \\
		      %                                                              & = \left[ -\tfrac{1}{6}(1-t)^6 \right]_{0}^{1} = \frac{1}{6}
		      %         \end{aligned}
		      %        $$
		      % \end{enumerate}
		      % $$
		      %  \int_C F \cdot dr
		      %  = \frac{1}{4} + \frac{1}{3} - \frac{3}{4} + \frac{1}{6}
		      %  = 0
		      % $$
		      $$
			      \begin{aligned}
				      \frac{\partial P}{\partial y} & = \frac{\partial}{\partial y} (x^3 + xy^2) = 2xy \\
				      \frac{\partial Q}{\partial x} & = \frac{\partial}{\partial } (x^2 y + y^5) = 2xy
			      \end{aligned}
		      $$
		      $\frac{\partial P}{\partial y} = \frac{\partial Q}{\partial x} \wedge \; \R^2$ is simply connected $ \wedge \; F$ is continuously differentiable \\
		      $\implies$ $F$ is conservative and $\int_C F \cdot dr = 0$.

		\item
		      $$
			      F \text{ conservative} \overset{\text{Theorem 1.4.15}}{\implies} F \text{ has a potential } \phi
		      $$
		      % TODO: How can you explain the result of part (a)?

		\item
		      We need to find $\phi$ such that:
		      $$
			      \frac{\partial \phi}{\partial x} = P = x^3 + x y^2, \quad \frac{\partial \phi}{\partial y} = Q = x^2 y - y^5.
		      $$

		      Integrate $P$ with respect to $x$:
		      $$
			      \phi(x,y) = \int (x^3 + x y^2)\, dx = \frac{1}{4}x^4 + \frac{1}{2}x^2 y^2 + h(y),
		      $$
		      where $h(y)$ is an unknown function of $y$ only.

		      $$
			      \frac{\partial \phi}{\partial y} = \frac{\partial}{\partial y} \left( \frac{1}{4}x^4 + \frac{1}{2}x^2 y^2 + h(y) \right) = x^2 y + h'(y).
		      $$

		      With $\frac{\partial \phi}{\partial y} = Q = x^2 y - y^5$:
		      $$
			      x^2 y + h'(y) = x^2 y - y^5 \implies h'(y) = -y^5
		      $$
		      $$
			      h(y) = \int -y^5\, dy = -\frac{1}{6} y^6 + C
		      $$

		      WLOG, take $C = 0$:
		      $$
			      \phi(x,y) = \frac{1}{4}x^4 + \frac{1}{2}x^2 y^2 - \frac{1}{6} y^6
		      $$

		      Let $\gamma = $ any path from $(0,0)$ to $(1,2)$. Since $F = \nabla \phi$, we have:
		      $$
			      \int_\gamma F \cdot dr = \phi(1,2) - \phi(0,0)
		      $$
		      with
		      $$
			      \begin{aligned}
				      \phi(1,2) & = \frac{1}{4} 1^4 + \frac{1}{2} 1^2 2^2 - \frac{1}{6} 2^6 = -\frac{101}{12} \\
				      \phi(0,0) & = 0 + 0 - 0 = 0
			      \end{aligned}
		      $$

		      Therefore:
		      $$
			      \int_{(0,0)}^{(1,2)} F \cdot dr = -\frac{101}{12}
		      $$
	\end{enumerate}
\end{solution}
